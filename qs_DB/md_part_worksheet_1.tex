【例4】 (22-23九年级上·龙江哈尔滨·阶段练习)如图,在四边形ABCD中,
\(\angle A B C = \angle D = 90 ^ { \circ }\) ,连接\(A C\) ,点F为边
\(C D\) 上点,连接BF交AC于点 \(E\) , \(A B = A E\)
,\(\angleFGC+\angleFBG=90^{\circ}\),
\(\angle B F G + 2 \angle G F C = 18 0 ^ { \circ }\) 若
\(A D = \frac { 7 \sqrt { 2 } } { 2 }\) \(B G = 4\) ,则CG的长为\_

\pandocbounded{\includegraphics[keepaspectratio]{../qs_image_DB/Snipaste_2025-10-21_17-34-02/760df7fced4ad0e15ecac4846939aedc2034f1ae6e9cb93ec79dec551bef9438.jpg}}
